\section{Navn}\label{sec:navn}
Prosjektets navn er Hackerspace NTNU; Hackerspace kan også benyttes.

Hackerspace er et prosjekt under Institutt for datateknologi og informatikk (IDI) ved NTNU stiftet 28.~november~2013.

Hackerspace sin logo kan kun endres av generalforsamling (\ref{sec:generalforsamling}); den øvrige grafiske profilen kan endres av styret (\ref{sec:struktur}). Den fullstendige grafiske profilen skal være tilgjengelig for medlemmer.


\section{Formål}\label{sec:formål}
Formålet til prosjektet er å fremme et tilgjengelig miljø for tverrfaglig teknologisk prosjektutvikling, samt vedlikeholde og drifte et synlig, inkluderende og kompetent teknologimiljø for alle studenter og ansatte på NTNU.\@


\section{Medlemskap}\label{sec:medlemskap}
\subsection{Medlemstyper}\label{sec:medlemskap:medlemstyper}
Et aktivt medlem er en student ved NTNU som har takket ja til medlemskap etter opptak (\ref{sec:medlemskap:opptak}) og ikke har avsluttet medlemskapet (\ref{sec:medlemskap:oppsigelse}~og~\ref{sec:medlemskap:utkastelse}).
Et aktivt medlem skal så langt det er mulig følge medlemspliktene (\ref{sec:medlemskap:medlemsplikter}).

Et pensjonert medlem -- en pang -- er et tidligere aktivt medlem som ikke har blitt kastet ut (\ref{sec:medlemskap:utkastelse}), som var aktivt medlem i minst 4 semestre eller styremedlem i 2 semestre (medregnet det man ble tatt opp i), og som har gjennomført Hackerspace Grunnkurs (\ref{sec:medlemskap:medlemsplikter}).
En pang har -- så lenge de er studenter ved NTNU -- fri tilgang til prosjektets lokaler, utstyr og kommunikasjonskanaler på linje med aktive medlemmer.
En pang har i tillegg alltid tilgang til de generelle interne kommunikasjonskanalene -- også etter endt studium.

\subsection{Opptak}\label{sec:medlemskap:opptak}
Styret har ansvar for å arrangere opptak; alle studenter ved NTNU kan søke.

Opptak skal som hovedregel avholdes hver høst. Ekstra opptak kan avholdes på våren om det er ønskelig.

Alle personopplysninger knyttet til søknadsprosessen er taushetsbelagte. Alle skriftlige dokumenter med personopplysninger skal destrueres innen én måned etter opptaksperioden.

\subsection{Medlemsplikter}\label{sec:medlemskap:medlemsplikter}
Aktive medlemmer skal tilegne seg nødvendig kunnskap til å holde verkstedet åpent.
Dette utgjør Hackerspace Grunnkurs og skal gjøres tilgjengelig av styret.

Aktive medlemmer har følgende faste plikter:
\begin{itemize}
\item Aktiv deltakelse i minst én gruppe (\ref{sec:struktur}).
\item Ukentlig deltakelse på verkstedets vaktliste.
\item Støtte prosjektet gjennom stand, kurs og lignende ved behov.
\end{itemize}
Ledelsen kan gi fritak fra plikter ved behov.

\subsection{Oppsigelse av medlemskap}\label{sec:medlemskap:oppsigelse}
Medlemmer kan når som helst avslutte medlemskapet sitt.
Dette skal skje skriftlig til enten nærmeste leder eller ledelsen.
Om medlemmet oppfyller kravene til å pange (\ref{sec:medlemskap:medlemstyper}), kan medlemmet velge å endre medlemsstatus til pang -- hvis ikke vil medlemmet avslutte medlemskapet komplett.

Om et medlem ikke deltar på noen aktiviteter over lenger tid og ikke er mulig å kontakte, antas det at medlemmet har avsluttet medlemskapet sitt.

\subsection{Utkastelse}\label{sec:medlemskap:utkastelse}
I særskilte situasjoner kan styret si opp enkeltpersoners medlemskap i Hackerspace.
Dette skal -- så langt det lar seg gjøre -- skje etter skriftlig advarsel og mulighet til endring av forholdene.

Et slikt vedtak skal overprøves av førstkommende generalforsamling om enten styret eller den utkastede ønsker det. Den som får medlemskapet oppsagt har rett til å kalle inn til ekstraordinær generalforsamling.


\section{Struktur}\label{sec:struktur}
\subsection{Grupper}\label{sec:struktur:grupper}
Hackerspace er delt inn i følgende grupper:
\begin{itemize}
\item Ledelsen
\item LabOps + PR (felles gruppe)
\item DevOps
\item Et varierende antall prosjektgrupper
\end{itemize}
Ledelsen utnevnes av generalforsamlingen (\ref{sec:generalforsamling}). De andre gruppene settes ned av styret (\ref{sec:struktur:styret}) og velger sin egen leder og nestleder internt.

\subsection{Styret}\label{sec:struktur:styret}
Styret består av hele ledelsen, samt leder og nestleder i hver av de øvrige gruppene.
Styret skal iverksette generalforsamlingens vedtak, kalle inn til generalforsamling, og utnevne de forskjellige gruppene med unntak av ledelsen.
Styret kan ta større avgjørelser og har mer makt enn ledelsen.
Styret skal godkjenne promotering og samarbeid med andre organisasjoner.
Styret kan vedta å gå over budsjett, men dette må meldes fra til medlemmer når det vedtas og overprøves av førstkommende generalforsamling.

Styrets møter er åpne, med mindre styret vedtar å lukke dem for enkeltsaker.
Styremøter kan kalles inn av ledelsen eller en tredjedel av styremedlemmene.
Styret plikter å annonsere tid og sted for styremøter i de alminnelige kommunikasjonskanalene til medlemmene, så snart det er fastsatt.
Dette bør annonseres minst 7 dager før møtet, og skal annonseres minst 24 timer før møtet.
Lukkede styremøter skal annonseres på samme vis sammen med grunn til lukking; møtested for lukkede møter trenger ikke oppgis.

Ved avstemninger i styret tildeles én stemme til hver av de statuttfestede gruppene, med unntak av prosjektgruppene.
Hver av prosjektgruppene tildeles en brøkdelsstemme tilsvarende $\frac{1}{\text{antall prosjektgrupper}}$, slik at prosjektgruppene til sammen har én stemme.
Om nødvendig, tildeles én vippestemme til økonomiansvarlig.

\subsection{Ledelsen}\label{sec:struktur:ledelsen}
Ledelsen består av leder, nestleder og økonomiansvarlig.
Disse stillingene velges ved personvalg (\ref{sec:valg}) av generalforsamlingen (\ref{sec:generalforsamling}).
Ledelsen står for den daglige driften av Hackerspace.

\subsection{LabOps + PR (felles gruppe)}\label{sec:struktur:labops}
Gruppen skal vedlikeholde verkstedet, arrangere kurs og bedrive PR for Hackerspace.

\subsection{DevOps}\label{sec:struktur:devops}
Gruppen skal utvikle og vedlikeholde Hackerspace sine IKT-løsninger.

\subsection{Prosjektgruppene}\label{sec:struktur:prosjektgruppene}
Prosjektgruppenes formål er å lage prosjekter for og om Hackerspace.

\subsection{Tillitsvalgt}\label{sec:struktur:tillitsvalgt}
Hackerspace skal ha en tillitsvalgt som blir utnevnt under generalforsamlingen. Tillitsvalgt kan ikke sitte i styret.
Denne tillitsvalgte har ingen stemme i styret, men er en kontaktperson for medlemmer av Hackerspace for saker de ikke ønsker å ta direkte opp med styret.
Som en del av dette arbeidet skal tillitsvalgt gjennomføre minst én trivselsundersøkelse i året blant de aktive medlemmene.
Om styret vedtar å lukke styremøter har tillitsvalgt fortsatt møte- og talerett.
Tillitsvalgt kan vedta å gi enkeltpersoner innsyn i referater fra lukkede styremøter.
Tillitsvalgt har taushetsplikt, og vil ved behov videreformidle saker anonymisert til styret.

Styret plikter å formidle tilbudet om tillitsvalgt til medlemmer av Hackerspace, behandle saker som blir tatt opp av tillitsvalgt, og respektere taushetsplikten til tillitsvalgt.

\subsection{Innsyn i dokumenter}\label{sec:struktur:innsyn}

Alle dokumenter i gruppene og styret skal, så sant de er av interesse for resten av Hackerspace og ikke er taushetsbelagte, være åpent tilgjengelig for innsyn for alle aktive medlemmer og panger som definert i paragraf~\ref{sec:medlemskap:medlemstyper}.



\section{Generalforsamling}\label{sec:generalforsamling}
\subsection{Om generalforsamling}\label{sec:generalforsamling:om}
Generalforsamlingen er Hackerspace sitt øverste organ og kan overstyre alle andre vedtak.
Det er kun generalforsamlingen som kan vedta budsjett, godkjenne regnskap, velge ledelse og endre statutter.
Generalforsamlingen møtes som regel én gang i året.

\subsection{Før generalforsamling}\label{sec:generalforsamling:før}
Generalforsamlingen kalles inn av styret direkte til alle medlemmer, med minst én måneds varsel.
Saker til generalforsamlingen skal sendes til styret minst to uker før generalforsamlingen, og endelig saksliste, fullt årsregnskap og budsjettforslag for det kommende året skal gjøres tilgjengelig for alle medlemmer minst én uke før generalforsamlingen.

\subsection{Under generalforsamling}\label{sec:generalforsamling:under}
Generalforsamlingen kan ved $\frac{2}{3}$ flertall av oppmøtte vedta å legge til saker på dagsorden, med unntak av statuttendringer.
Statuttendringer kan kun behandles om forslag er sendt ut sammen med siste innkalling minst én uke før.
Statuttendringer kan kun vedtas ved $\frac{2}{3}$ flertall og 40\% av prosjektets stemmer avlagt.
Det er ikke tillatt å stemme blankt på en statuttendring; det er dog tillatt å avstå fra å stemme.

Alle medlemmer -- både aktive og panger -- har møte- og talerett.
Aktive medlemmer har stemmerett.
Stemmeretten kan gis videre ved skriftlig fullmakt; denne kontrolleres av tellekorpset på generalforsamlingen.

\subsection{Etter generalforsamling}\label{sec:generalforsamling:etter}
Den nye ledelsen trer i kraft to uker etter generalforsamlingen.
Alle andre grupper skal avholde valg av leder og nestleder ved første anledning etter generalforsamlingen.

\subsection{Ekstraordinær generalforsamling}\label{sec:generalforsamling:ekstraordinær}
Det skal kalles inn til ekstraordinær generalforsamling om styret eller minst $\frac{1}{3}$ av de aktive medlemmene krever det.
Innkallingen sendes da ut minst to uker i forkant sammen med saksliste.


\section{Valg}\label{sec:valg}
Alle personvalg med flere kandidater skal være anonyme.
Stemmeopptellingen foregår lukket av valgkomiteen, med mindre $\frac{2}{3}$ av de stemmeberettigede som deltar ved valget ønsker noe annet.

Personvalg gjennomføres ved at personen med minst 50\% av stemmene vinner valget.
Er det flere kandidater hvor ingen har flertall, skal kandidaten med minst stemmer fjernes fra listen, og en ny runde med stemming avholdes.
Dette gjentas til én kandidat får minst 50\% av stemmene.
Skulle det være to kandidater igjen -- hvor begge har fått et likt antall stemmer, holdes en ny runde avstemning.

Ledelsen samt de andre gruppenes leder og nestleder velges i utgangspunktet for ett år av gangen, og kan ikke gjenvelges etter to år i det samme vervet om det finnes andre kandidater.
Tidligere leder i ledelsen eller en annen gruppe kan ikke velges til nestleder i den samme gruppen om det finnes andre kandidater.

Ingen avstemninger skal holdes slik at noen får en urettmessig fordel.


\section{Statuttendringer}\label{sec:statuttendringer}
Styret kan endre oppsett, utforming, formulering og skrivefeil i statuttene, såfremt betydningen bevares.
Styret må redegjøre for endringene offentlig gjennom prosjektets interne kommunikasjonskanaler overfor alle medlemmene.
Om et medlem protesterer endringen, må den behandles på neste generalforsamling.
Om ingen medlemmer protesterer, trer endringen i kraft to uker etter offentliggjøringen.


\section{Oppløsning}\label{sec:oppløsning}
Prosjektet kan oppløses dersom det vedtas på generalforsamlingen.
Dette krever flertall på minst $\frac{2}{3}$ av prosjektets medlemmer.
Ved avvikling skal det velges et avviklingsstyre som håndterer oppløsning i samarbeid med IDI.\@
Generalforsamlingen fatter vedtak om hvordan prosjektets midler skal disponeres av avviklingsstyret etter oppløsningen.
